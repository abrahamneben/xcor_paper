%% This is a sample manuscript marked up using the
%% AASTeX v5.x LaTeX 2e macros.

%% The first piece of markup in an AASTeX v5.x document
%% is the \documentclass command. LaTeX will ignore
%% any data that comes before this command.

%% The command below calls the preprint style
%% which will produce a one-column, single-spaced document.
%% Examples of commands for other substyles follow. Use
%% whichever is most appropriate for your purposes.
%%

%\documentclass[11pt,preprint2]{aastex}
% manuscript produces a one-column, double-spaced document:
%\documentclass[manuscript]{aastex}
%% preprint2 produces a double-column, single-spaced document:

\documentclass{emulateapj}
%\documentclass[preprint]{aastex}

\usepackage{natbib}
\usepackage{amsmath}
\bibliographystyle{apj}
\usepackage{graphicx}

\usepackage{color}
\usepackage[normalem]{ulem}
\usepackage{url}

\DeclareUrlCommand\ULurl{%
  \renewcommand\UrlFont{\ttfamily\color{blue}}%
  \renewcommand\UrlLeft{\bgroup}%
  \renewcommand\UrlRight{\egroup}}


\newcommand{\mintext}{\text{min}}
\newcommand{\maxtext}{\text{max}}
\newcommand{\rad}{\text{rad}}
\newcommand{\IR}{\text{IR}}
\newcommand{\ir}{\text{IR}}
\newcommand{\therm}{\text{therm}}
\newcommand{\cosmo}{\text{cosmo}}
\newcommand{\fg}{\text{fg}}
\newcommand{\res}{\text{res}}
\newcommand{\shot}{\text{shot}}
\newcommand{\SNR}{\text{SNR}}

\newcommand{\Fb}{\mathbf{F}}
\newcommand{\Mb}{\mathbf{M}}
\newcommand{\Cb}{\mathbf{C}}
\newcommand{\Ab}{\mathbf{A}}
\newcommand{\xb}{\mathbf{x}}
\newcommand{\pb}{\mathbf{p}}
\newcommand{\qb}{\mathbf{q}}
\newcommand{\Wb}{\mathbf{W}}


%% Sometimes a paper's abstract is too long to fit on the
%% title page in preprint2 mode. When that is the case,
%% use the longabstract style option.

%% \documentclass[preprint2,longabstract]{aastex}

%% If you want to create your own macros, you can do so
%% using \newcommand. Your macros should appear before
%% the \begin{document} command.
%%
%% If you are submitting to a journal that translates manuscripts
%% into SGML, you need to follow certain guidelines when preparing
%% your macros. See the AASTeX v5.x Author Guide
%% for information.

%% You can insert a short comment on the title page using the command below.

%\slugcomment{Not to appear in Nonlearned J., 45.}

%% If you wish, you may supply running head information, although
%% this information may be modified by the editorial offices.
%% The left head contains a list of authors,
%% usually a maximum of three (otherwise use et al.).  The right
%% head is a modified title of up to roughly 44 characters.
%% Running heads will not print in the manuscript style.

\shorttitle{Foreground and sensitivity analysis for 21cm/Infrared studies}
\shortauthors{Neben et al.}

%% This is the end of the preamble.  Indicate the beginning of the
%% paper itself with \begin{document}.



\begin{document}

\title{Foreground and sensitivity analysis for broad band 21\,cm--Ly$\alpha$ and $21\,cm$--H$\alpha$ correlation experiments probing the Epoch of Reionization}

%% Use \author, \affil, and the \and command to format
%% author and affiliation information.
%% Note that \email has replaced the old \authoremail command
%% from AASTeX v4.0. You can use \email to mark an email address
%% anywhere in the paper, not just in the front matter.
%% As in the title, use \\ to force line breaks.

\author{Abraham R. Neben\altaffilmark{1},
Brian Stalder\altaffilmark{2},
John L. Tonry\altaffilmark{2},
Jacqueline N. Hewitt\altaffilmark{1}}

\affil{\altaffilmark{1}MIT Kavli Institute, Massachusetts Institute of Technology, Cambridge, MA, 02139 USA}
\affil{\altaffilmark{2}Institute for Astronomy, University of Hawaii, 2680 Woodlawn Drive, Honolulu HI 96822}

%% Notice that each of these authors has alternate affiliations, which
%% are identified by the \altaffilmark after each name.  Specify alternate
%% affiliation information with \altaffiltext, with one command per each
%% affiliation.

%% Mark off your abstract in the ``abstract'' environment. In the manuscript
%% style, abstract will output a Received/Accepted line after the
%% title and affiliation information. No date will appear since the author
%% does not have this information. The dates will be filled in by the
%% editorial office after submission.

\begin{abstract}
aoeu
\end{abstract}

\keywords{cosmology: observations --- dark ages, reionization, first stars --- infrared: diffuse background}

\section{Introduction}

Deep radio and infrared observations are nearing detection of the first stars and galaxies from the cosmic dawn. As such sources form, they are thought to blow out ionized bubbles, eventually merging and reionizing the universe \citep{FurlanettoReview,miguelreview,PritchardLoebReview}. First generation 21\,cm observatories such the Murchison Widefield Array (MWA) \citep{tingay13,mwascience} and the Precision Array for Probing the Epoch of Reionization (PAPER) \citep{parsons14,ali15,PoberPAPER64Heating,DannyMultiRedshift} are setting ever tighter limits on redshifted neutral hydrogen emission from the neutral regions between these bubbles, and the now-underway Hydrogen Epoch of Reionization Array (HERA) \citep{deboer16} is expected to detect and characterize the EOR power spectrum in the coming years. Ultimately, the Square Kilometer Array (SKA) will image the EOR over redshift, revealing the detailed hydrogen reionization history of the universe \citep{ska}. 

At the same time, deeper galaxy redshift surveys are beginning to constrain the reionizing sources themselves. Hubble deep field observations \citep{Bouwens2011,Illingworth2013,Dunlop2013} and cluster lensing surveys are finding tens of galaxies at $6<z<10$ down to UV magnitudes of $M_{AB}\sim-17$, and extremely wide surveys are searching for the rare bright ones \citep{Schmidt2014,Trenti2011,Bradley2012}. However, current models require the contribution of far fainter galaxies down to $M_{AB}\sim-13$ \citep{Robertson2013,Alvarez2012} in order to agree with optical depth measurements \citep{planck16} from the cosmic microwave background. Deeper observations with the James Webb Space Telescope (JWST) \citep{Gardner2006} and the Wide Field Infrared Space Telescope (WFIRST) \citep{Spergel2013} will be needed to probe this crucial faint population \citep{Atek2015}.

Infrared intensity mapping offers several advantages compared to surveys. Power spectrum analyses can be sensitive to an EOR component even if the signal-to-noise in individual pixels small, and instead of being limited to the brightst galaxies, intensity mapping is sensitive to the cumulative light from \textit{all} sources. Indeed,  ionizing and Lyman-alpha radiation from EOR galaxies at $z\sim6-8$ redshifts into the near infrared, motivating intensity mapping at micron-scale wavelengths. Working around foregrounds is challenging, though. While early studies suggested angular fluctuations in infrared intensity maps traced EOR galaxies \citep[e.g.,][]{kash1,kash2,kash3}, \citet{kash4} find that given current constraints on the EOR, this is unlikely. In fact, \citet{cooray12,zemcov14} argue that intrahalo light, consisting of tidally stripped stars dispersed throughout host halos, is the best explanation for the observed fluctuation excess over known EOR galaxy populations. This implies that even after significant foreground masking, EOR foreground emission in wide field infrared surveys is of order $10^4$ times brighter than EOR emission at $1'-60'$ scales.

Given these bright foregrounds, cross correlation with 21\,cm maps may in fact be the \textit{only} way to extract the diffuse EOR component of the near infrared background. The synergy is clear: the galaxies sourcing reionization generate strong Ly-$\alpha$ emission, while the neutral regions between them glow at rest frame 21\,cm. On typical ionized bubble scales, bright spots in IR maps likely correspond to ionized regions, and thus, 21\,cm dark spots, and vice versa, sourcing an anticorrelation seen in EOR simulations by \citet{silva12,heneka16}. 

A similar anticorrelation on large scales is found by \citet{lidz09,park14} in simulations of 21\,cm cross correlation with galaxy surveys, but conducting redshift surveys both wide and deep enough to cross correlate with 21\,cm maps is challenging due to the hugely different spatial scales probed by 21\,cm experiments and spectroscopic galaxy surveys. For instance, the 3' angular resolution of the MWA is of roughly the field of view of the Hubble Deep Field and the James Webb Space Telescope (JWST). It may be possible to cross-check the ionization environment of deep JWST sources by comparing the brightness temperature in the 21\,cm map \citep{beardsley15}, but even after order $\sim100$\,hour integrations such detections will be near JWST limiting sensitivities \citet{zackrisson11}.

In contrast, broad band intensity enables similar science with shallower observations \citep{StarsAndReionization,mao14}, though imperfect radio and infrared foreground subtraction will leak largely uncorrelated power into the cross correlation analysis which must be averaged out over large fields of view. Fortunately, the planned Transiting Exoplanet Survey Satellite \citep{ricker14} and the proposed SPHEREx satellite mission \citep{ScienceWithSpherex,SpherexWhitePaper} would image the entire sky in the near infrared, and many ground-based near infrared surveys with few degree fields such as the Dark Energy Survey \citep{des16}, Pan-STARRS \citep{tonry12}, and the Asteroid Terrestrial-impact Last Alert System (ATLAS) \citep{tonry11} are coming online. In the low frequency radio, the MWA has performed a deep survey of 400 square degrees at high galactic latitude \citep{beardsley16}, and HERA will survey $\sim2000$ square degrees along a zenith strip \citep{dillonmapmaking}. It is important to note that a large, uniform focal plane greatly facilitates intensity mapping, lest structures on relevant angular scales be lost in the calibration of many independent regions of a segmented focal plane, such as that of Pan-STARRS.

With wide and deep near infrared and low frequency radio surveys happening now 
and on the horizon, we study in this paper the real world prospects of detecting the
 anti-correlation of diffuse 21\,cm, Ly$\alpha$, and H$\alpha$ emission from the EOR.
  We begin in Sec. 2 with a review of our fourier transform and power spectrum conventions.
  In Sec. 3 we present the MWA and ATLAS observations we use and discuss processing these data into images. 
   In Sec. 4 we characterize the bright radio and infrared point source foregrounds in such measurements, 
   demonstrating that distance variation of the sources combined with
   their finite luminosity distribution introduces slight positive correlations which overpower the cosmological signal
   unless significant masking and subtraction are done. 
   In Sec. 5, we study how best to mask and subtract radio and infrared foregrounds on real world datasets and 
   quantify the foreground residuals with their power spectra. We set the first limits 
   on the broad band  21\,cm--Ly$\alpha$ cross spectrum at $z=7$ using data from the MWA
     and ATLAS, and compare the sensitivities of future experiments, 
     illustrating what it will take to realize this measurement.

\section{Power spectrum and correlation conventions}
\label{sec:pspecconventions}

\subsection{Power spectrum definitions}

We define the 3D power spectrum $P(\vec{k})$ of the image cube $I(x,y,z)$ following \citet{ewallwice14} as 
\begin{equation}
	P(\vec{k}) = \frac{\langle|\tilde{I}(\vec{k})|^2\rangle}{V}
\end{equation}
where
\begin{equation}
	\tilde{I}(\vec{k})=dV\sum_{\vec{x}}I(\vec{x})e^{i\vec{k}\cdot\vec{x}}
\end{equation}
and $V$ is the survey volume, and $dV$ is the voxel size.

Similarly, over narrow fields of view, the angular power spectrum $C_\ell$ of a 2D (e.g, broad band) image $I(\vec{\theta})$ can be shown to be approximately
\begin{equation}
\label{eqn:Cldef0}
	C(\vec{\ell}) = \frac{\langle|\tilde{I}(\vec{\ell})|^2\rangle}{\Omega} 
\end{equation}
where
\begin{equation}
	\tilde{I}(\vec{\ell})=d\Omega\sum_{\vec{\theta}}I(\vec{\theta})e^{i\vec{\ell}\cdot\vec{\theta}}
\end{equation}
and $\Omega$ is the survey solid angle, and $d\Omega$ is the pixel size. Thus, we need only evaluate a fourier transform to estimate $C_\ell$ over a narrow field of view. Writing this out in detail, we find\footnote{IS IT REALLY RIGHT THAT ALL THESE PEOPLE WROTE IT UP WRONG? ARE THERE DIFFERENT NORMALIZATIONS OF C$\ell$ ? Note that the normalization of $d\theta^2/N^2$ has been misstated as $1/N^2$ by \citet{zemcov14} (Eqn. 3 of their supplement), $d\theta$ by \citet{cooray12} (Eqn. 1 of their supplement), and $N^2d\ell^2/(2\pi)^2$ by \citet{thacker15} (Eqn. B3). Using Eqn. \ref{eqn:elldef}, the last form can be shown equal $1/d\theta^2$.} 
\begin{equation}
\label{eqn:Cldef}
	C(\ell(a,b))=\left\langle\left|\sum_{m,n}I(m,n)\exp\left(-\frac{2\pi i}{N}  (am+bn)\right)\right|^2\right\rangle\frac{d\theta^2}{N^2}
\end{equation}
where $d\theta=d\theta_x=d\theta_y$ is the pixel size, $N\equiv N_x=N_y$ is number of pixels on a side of a square image, and $\ell=\sqrt{\ell_x^2+\ell_y^2}$, where 
\begin{eqnarray}
\ell_x&=&2\pi a/N d\theta \label{eqn:elldef}\\
\ell_y&=&2\pi b/Nd\theta \label{eqn:elldef2}
\end{eqnarray}

Note that $C_\ell$ has the units of $[I^2d\theta^2]$. We will generally work with the dimensionally more intuitive power spectrum $\Delta(\ell)$, given by
\begin{equation}
	\Delta(\ell)=\sqrt{\frac{\ell^2}{2\pi}C(\ell)}
\end{equation}
which has the same units as I because $\ell$ has units of 1/rad. 

The 3D power spectrum is often cylindrically binned to $(k_\perp,k_\parallel)$ space where $k_\perp^2\equiv k_x^2+k_y^2$ represents modes perpendicular to the line of sight, and $k_\parallel=k_z$ represents modes along the line of sight. We show in Appendix \ref{sec:pspecrelation}  that this cylindrically binned power spectrum is related to the angular power spectrum of a broad band image (over a narrow field of view) as
\begin{equation}
P(k_\perp,k_\parallel=0)=D_c^2 \Delta D_c C_{\ell(k_\perp)}	
\end{equation}
with $\ell=D_c k_\perp$, where $D_c$ is the comoving line of sight distance to the center of the cube, and $\Delta D_c$ is the comoving depth of the cube.

\subsection{Cross spectrum vs. coherence}

We extend the definition of the angular power spectrum given in Eqn. \ref{eqn:Cldef0} to the cross spectrum as
\begin{equation}
\label{eqn:Cldefcross}
	C_{A,B}(\vec{\ell}) = \frac{\langle \tilde{I}_A^*(\vec{\ell})\tilde{I}_B(\vec{\ell})\rangle}{\Omega}
\end{equation}
where $A$ and $B$ denote the 21\,cm and the IR fields. The cross spectrum is a quantity which ranges between $\pm(C_{A}(\vec{\ell})C_{B}(\vec{\ell}))^{1/2}$ depending on how correlated or anti-correlated the two fields are. 

The cross spectrum is thus often renormalized as  
\begin{equation}
\label{eqn:Cldefcross}
	c_{A,B}(\vec{\ell}) \equiv \frac{C_{A,B}(\vec{\ell}) }{\sqrt{C_A(\vec{\ell})  C_B(\vec{\ell}) }}
\end{equation}
where $c$ is known as the coherence. The coherence quantifies the fractional correlation of the two fields, and is insensitive to a simple rescaling of either one. However, large foreground residuals afflicting either field will substantially bias the coherence towards zero \citep{lidz09,furlanettolidz07}. 

\section{Observations and Imaging}
\subsection{21\,cm Observations}
\label{sec:mwaobservations}

The MWA is a low frequency radio interferometer in Western Australia consisting of 128 phased array tiles, each with $\sim30^\circ\times(150\text{\,MHz}/f)$ field of view and steerable in few degree increments with a delay line beamformer. We use MWA low frequency radio observations of a quiet field centered at (RA,Dec)=($0^\circ$,$-27^\circ$) J2000 recorded over 30.72\,MHz bandwidth around 186\,MHz, corresponding to $z=6.0-7.3$. We begin with the 32 hours of observations calibrated, foreground-subtracted, and imaged by \citet{beardsley16}. These 32 hours result from applying stringent quality control cuts to roughly 80 hours of observations recorded between August and November 2013, where spectrally narrow RFI is flagged using COTTER \citep{AndreMWARFI}, based on AOFlagger\footnote{\ULurl{http://aoflagger.sourceforge.net}} \citep{aoflagger} and spectrally broad RFI is flagged by comparing the spectral smoothness of the data compared to baseline levels. 

The 32 hours of observations are processed separately by \citet{beardsley16} as in 2 minute snapshots, and imaging is conducted using Fast Holographic Deconvolution\footnote{\ULurl{https://github.com/EoRImaging/FHD}} \citep{fhd}. Model visibilities are predicted from a foreground model of diffuse and point source emission in the field \citep{beardsley16,PattiCatalog1}, to be used for both calibration and foreground subtraction. For each snapshot, FHD produces naturally weighted data and model image cubes as well as primary and synthesized beam cubes. We flag the upper and lower 80\,kHz channels in each of 24 coarse channels across the band to mitigate aliasing, and average the remaining channels in frequency to make a broad band image. 

FHD outputs these ``cubes'' in HEALPix format per frequency. Note that this processing is performed in parallel on ``odd'' and ``even'' data cubes whose data are interleaved in time to allow more leverage on estimating the magnitude of the noise and the noise bias. We average the cubes in frequency, rotate them so the EOR0 field center lies at the north pole, and project the pixels into the $xy$ plane, resulting in an orthographic projection to the plane tangent to  $(\text{RA}, \text{Dec})=(0,-27^\circ)$.


\subsection{IR Observations}

\begin{figure}[h]
\centering
\includegraphics[width=3.3in]{images/survey_overview.pdf}
\caption{5' pixels}
\label{fig:surveyoverview}
\end{figure}

ATLAS is a 0.5\,m (f/2.0) telescope \citep{tonry11} in Hawaii designed to perform a wide field sky survey for near earth asteroids. The detector is a 10,500 $\times$10,500 STA-1600 CCD array, with a pixel scale of 1.86'', with an overall field of view of $5.5^\circ$. We observe in the I band centered at 807\,nm with full width at half max 150\,nm, corresponding to $z=$5.1--6.3. While this redshift range doesn't exactly match that of our r radio observations, it overlaps sufficiently for our purpose of characterizing the effects of noise foregrounds in 21cm--IR cross correlation experiments. 

We perform two separate observing campaigns, which we illustrate in Fig. \ref{fig:surveyoverview}. We first perform a wide survey to best characterize bright foregrounds; . In the first campaign we raster scan a roughly $20^\circ\times20^\circ$ grid with $5^\circ$ spacing over the MWA field (dashed black circle), integrating for 2.5\,min at each pointing (blue square markers). The observations are conducted between 2016/09/07 22:00  and 2016/09/08 00:50 Hawaii-Aleutian Standard Time, when the moon was 36\% illuminated.  Later we will find that this is substantially more integration time than required to mitigate photon shot noise, but not enough to mitigate diffuse airglow structure and CCD array systematics. Still, it is sufficient for performing a source survey where source fluxes are estimated relative to the local background. 

We then use {\tt swarp}\footnote{\ULurl{http://www.astromatic.net/software/swarp}} \citep{swarp} to stack all these frames over $20^\circ$ orthographic field centered on (RA,Dec) = $(0^\circ,-30^\circ)$ (blue square) with 1.86'' resolution. We use the default background subtraction settings to mitigate temporal and spatial background variation. 

% moon phase calculator: http://www.moonpage.com/index.html?go=T&auto_dst=T&tzone=ut&m=9&d=8&y=2016&hour=0&min=51&sec=45

Our second campaign is a deep survey designed to better mitigate airglow fluctuations and CCD systematics for the purpose of studying faint foregrounds. We select four 5 deg fields positioned around the MWA beam peak for best cross correlation precision: (RA,Dec) = $(-2.5^\circ, -24.5^\circ)$, $(2.5^\circ, -24.5^\circ)$, $(-2.5^\circ, -29.5^\circ)$, $(2.5^\circ, -29.5^\circ)$ (J2000). We raster scan a $3\times3$ grid of 30\,sec observations within each field (red circle markers) intended to mitigate slight amplifier non-uniformities across the CCD array. The observations were conducted on 2016/11/02 between 21:47 and 23:11 Hawaii-Aleutian Standard Time, when the moon was 5\% illuminated.

We stack the frames in each of the four deep fields using {\tt swarp}, saving only the central $4^\circ\times 4^\circ$ region over which all nine 30\,sec frames overlap (red squares). Otherwise slight background discontinuities would be introduced by the different temporal coverage of different regions of the stack. In this stacking, we disable background subtraction for the purpose of studying later on the effects of airglow-induced diffuse backgrounds.




\section{Point source foregrounds}

\subsection{Catalogs}
\label{sec:catalogs}

To characterize the bright sources relevant to 21\,cm--Ly$\alpha$ and  21\,cm--H$\alpha$ intensity mapping correlation measurements we calculate the correlations between catalogs at 185\,MHz, 850\,nm, and 4.5\,um as a function of mask depth. These bands correspond roughly to 21\,cm, Ly$\alpha$, and H$\alpha$ at $z\sim6-7$. 

\begin{figure*}[t]
\centering
\includegraphics[width=6.5in]{images/catalog_histograms.pdf}
\caption{5' pixels}
\label{fig:cataloghistograms}
\end{figure*}

We use the 185\,MHz catalog reduced from deep observations of the MWA field depicted in Fig. \ref{fig:surveyoverview} by \citep{PattiCatalog1} (left panel in Fig. \ref{fig:cataloghistograms}). Unlike the other catalogs we consider, the survey depth varies somewhat over the MWA field due to the varying primary beam, though within the full-width-at-half-max it is complete to roughly 100\,mJy. MWA astrometry is at the 2--3' level due to the MWA's short baselines, though \citep{PattiCatalog1} cross-match MWA detections with higher frequency catalogs to achieve order 10'' astrometry. 

We use the W2 band of ALLWISE \citep{Wright2010,allwise} as our 4.5\,um catalog. We download the list of sources within the MWA field using the All Sky Search on the NASA/IPAC Infrared Science Archive\footnote{\ULurl{http://irsa.ipac.caltech.edu}} (center panel in Fig. \ref{fig:cataloghistograms}). This ALLWISE band is specified to be 95\% complete to 88\,$\mu$Jy (15.7 AB mag), though it has slight sky coverage non-uniformities due to satellite coverage. We will see that we run into sky coverage problems only after masking the brightest 90\% of sources.

Lastly, we run SExtractor\footnote{\ULurl{http://www.astromatic.net/software/sextractor}} \citep{sextractor} on our wide $20^\circ$ wide ATLAS composite image to generate an 850\,nm catalog. We allow local background bias and noise estimation, and set pixel saturation at 20,000 counts to avoid artifacts. We find the catalog is complete to roughly a mJy (right panel in Fig. \ref{fig:cataloghistograms}). 


\subsection{Catalog radio--infrared flux correlations}
\label{sec:catcorrelations}

Having prepared catalogs of point source foregrounds in our three bands, we proceed to study how they manifest in intensity mapping correlation experiments. Traditionally, radio/optical/infrared correlations have been studied by cross matching high frequency radio detections with optical and infrared sources coincident within a few arcseconds, and plotting radio versus optical or infrared luminosity. Such studies have revealed the well known radio--far infrared correlation (CITATIONS) thought to be due to massive star formation \citep[e.g.][]{helou85,dejong85,yun01,xu94}. Such stars emit ionizing radiation which is reprocessed into far-infrared by dust, and also create HII regions which emit radio synchrotron \citep{xu94}. A secondary radio mechanism is synchrotron radiation from acceleration of cosmic ray electrons in these stars' supernovae. 

Our approach is different. For all the advantages of broad band intensity mapping, sources cannot be localized to specific redshifts, meaning that it is foreground fluxes, not luminosities, whose correlations are of interest. Of course compact foregrounds may be masked or subtracted to some residual level, but any correlation of these residual foreground fluxes could mask the cosmological correlation. We begin in this section by analyzing foreground fluxes as a function of masking depth, and in the next section turn to the foregrounds in residual images below the detection limit of these catalogs. A last comment on our approach is that though searching for radio--infrared correlations on a source-by-source level would be valuable cross check, we lack the radio astrometry to do so\footnote{In their cross matching study of GHz radio sources with optical detections, \citep{mcmahon02} find that 99\% of cross-matches within 10'' of each other (see their Fig. 8), of order the positional accuracy quoted by \citet{PattiCatalog1}.}. 

\begin{figure*}[h]
\centering
\includegraphics[width=5.5in]{images/source_correlation_grids_and_snrs.pdf}
\caption{5' pixels}
\label{fig:correlationsandSNRs}
\end{figure*}

We begin by gridding all three catalog fluxes in Jy to the $20^\circ\times20^\circ$ grid centered at (RA,Dec) = $(0, -30^\circ)$ depicted in Fig. \ref{fig:surveyoverview}, and calculating the zero delay correlations between the images $I_i(\theta_x,\theta_y)$ as
\begin{equation}
\label{eqn:imagecorrdef}
	c = \frac{\langle I_1I_2\rangle-\langle I_1\rangle\langle I_2\rangle}{\sqrt{(\langle I_1^2\rangle -\langle I_1\rangle^2)(\langle I_2^2\rangle -\langle I_2\rangle^2)}}
\end{equation}
where the sample variance in the case of small or zero correlation is dominated by the numerator and is given by $1/\sqrt{N_\text{pix}}$, where $N_\text{pix}$ is the total number of pixels in the image. Between MWA and WISE catalogs we find $c=-0.003\pm0.005$ and between MWA and ATLAS catalogs we find $c=0.0013\pm0.005$. Both are consistent with zero, which is not necessarily unexpected. The brightest sources in both infrared catalogs are likely stars, whose radio emission is vanishingly small. To illustrate, we calculate the correlations after excluding the brightest 10\% of sources in all three catalogs, and find $c_\text{MWA--WISE}=0.031\pm0.005$ and $c_\text{MWA--ATLAS}=0.0086\pm0.005$. The former is a $6\sigma$ detection, and merits some investigation. How does this apparent correlation depend on the flux cut? What is it due to? And what does it mean for broad band correlation experiments? Further, does the MWA--ATLAS correlation remain consistent with zero at stricter flux cuts?

To begin to answer these questions, we plot in Fig. \ref{fig:correlationsandSNRs} the 185\,MHz--4.5\,$\mu$m correlation (top left) and 185\,MHz--850\,nm correlation (top right) as a function of the maximum flux percentile not excluded from the data. For instance, the correlations we calculated above without any source masking are in the 100\%--100\% bin, while the correlation after removing the brightest 90\% of sources in both catalogs are in the 10\%--10\% bin. We indicate the fluxes that these percentiles correspond to in Fig. \ref{fig:cataloghistograms}. We plot the SNRs of these correlation measurements, taking the noise to be $1/\sqrt{N_\text{pix}}$ as described above, in the next row. Note that adjacent cells in the correlation matrix plots are strongly correlated, so a consistent positive sign is not in and of itself evidence of significance. We assess significance by comparing of each correlation measurement with the expected noise (the SNR), by checking that the correlation vanishes when the 185\,MHz image is flipped (bottom two rows).

The 185\,MHz and 4.5\,$\mu$m catalogs exhibit a positive correlation peaking at $0.0337\pm0.005$ after masking the brightest 10\% of 4.5\,$\mu$m sources (down to 18.25 mag) and the brightest 30\% of radio sources (down to 0.54\,Jy). The correlation is detected at greater than $3\sigma$ significance until either radio sources are masked down to the 0.2\,Jy level or 4.5\,$\mu$m sources are masked down to (19.8 mag) level. These levels are near the completeness limits of both catalogs, though, and even if the correlation extended to fainter fluxes is would not likely be significantly detected. There is no significant correlation detection after flipping the 185\,MHz image, indicating this is not an artifact of the analysis. The 185\,MHz and 850\,nm catalogs exhibit a smaller correlation, but one which nonetheless peaks at 3--4$\sigma$ significance over a broad range of flux cuts. This correlation peaks at $0.023\pm0.005$ after masking the brightest 30\% of 185\,MHz sources (down to 0.2\,Jy) and the brightest 70\% of 850\,nm sources (down to 17.9 mag). As before, this correlation vanishes as expected when the 185\,MHz image is flipped prior to cross correlation.

To understand these findings, we begin by digging deeper into the 4.5\,$\mu$m catalog. We select the subset of sources detected in WISE 3.4\,$\mu$m, 4.5\,$\mu$m, and 12\,$\mu$m bands and plot them (Fig. \ref{fig:wisecolorcolor}, left panel) in the $W_{23}\equiv$[4.6\,$\mu$m] -- [12\,$\mu$m] versus $W_{12}\equiv$[3.4\,$\mu$m] -- [4.6\,$\mu$m] color-color space used by \citet{Wright2010} to illustrate the separation of stars and various types of non-active and active galaxies. \citet{nikutta14} study how these different sources separate more quantitatively, finding that stars separate clearly from all other sources in the region $W{12}=-0.04\pm0.03$, $W_{23}=0.05\pm0.04$ (1$\sigma$). In the right panel, we plot the faintest 90\% of sources (fainter than 18.25 mags at 4.6\,$\mu$m) in the same color-color space and observe that all stars are excluded. This explains why before masking this brightest 10\%, we observed zero correlation with 185\,MHz sources, and why afterward the correlation suddenly emerges, after all, nearly all radio sources are non-stellar (i.e., extragalactic). 

\begin{figure*}[h]
\centering
\includegraphics[width=6in]{images/wise_color_color_figure.pdf}
\caption{5' pixels}
\label{fig:wisecolorcolor}
\end{figure*}

To further probe which infrared sources are responsible for this correlation, we make a simple cut to separate quasars (QSOs) and active galactic nuclei (AGNs) ($W_{12}>0.16$) from luminous red galaxies (LRGs) and starburst galaxies (SFs) ($W_{12}<0.16$) \citep{nikutta14,kurcz16}. In Fig. \ref{fig:wisexspec} we plot the power spectrum of 185\,MHz sources (left panel), the spectra of our QSO/AGN and LRG/SF cuts of 4.5\,$\mu$m sources (center panel), and the coherence (i.e., correlation coefficient versus $\ell$) of each of these cuts with the 185\,MHz catalog (right panel). The QSO/AGN subset exhibits no significant correlation with the 185\,MHz sources, while the LRG/SF cut exhibits a significant correlation rising from a few percent at $\ell\sim7000$ to 13\% at $\ell\sim300$. The falloff towards high $\ell$ is due to the 2' resolution of the MWA synthesized beam at 185\,MHz, corresponding to a maximum $\ell$ of roughly 6000. 

\begin{figure*}[h]
\centering
\includegraphics[width=7in]{images/mwa_wise_qsoagn_gal_xspec.pdf}
\caption{5' pixels}
\label{fig:wisexspec}
\end{figure*}

%argue that LRGs are negligible in terms of radio emission compared to LRGs

The fall of the correlation towards high $\ell$ is due to the similar fall in the 4.5\,$\mu$m catalog power spectrum towards high $\ell$. Both the falling 4.5\,$\mu$m catalog power spectrum and the relatively  which are complex functions of the survey properties. \citet{tegmark02,dodelson02} show that the galaxy angular power spectrum $C_\ell$ is approximately equal to the 3D matter power spectrum $P(k(\ell))$ convolved with with a window function which depends on the redshift coverage and flux limit of the sample. The matter power spectrum is known to rise as $k^{1}$ for $k\lesssim0.02$\,h/Mpc, above which it falls as $k^{-3}$. Galaxy surveys typically probe the regime just above the turnover where the slope is transitioning from 0 to -3 \citep{tegmark02b}. In order to maximize its sensitivity to low surface EOR 21\,cm emission, the MWA was designed as a relatively compact array in comparison to higher resolution radio interferometers such as the Very Large Array. This low resolution makes the MWA catalog severely flux limited \citep{PattiCatalog1}, which in turn effectively masks many galaxies which would otherwise be seen. This large spatial mask translates into a wide fourier convolution kernel, explaining why the MWA catalog power spectrum is so flat.

\subsection{Simulations of distance-induced flux correlation}

Let us now turn to addressing why the 4.5\,$\mu$mu LRG/SF sample is 5--15\% correlated with the 185\,MHz catalog, while the QSO/AGN sample is not. Of course some slight correlation is expected at some level as brighter AGN are typically in more massive galaxies, which are typically brighter in stars as well (see, for instance, Fig. 1 in \citep{seymour07} or Fig. 4 in \citep{Willott03}) however \citet{mauch07} find no evidence of a radio/near-infrared luminosity correlation anywhere near as strong as that between radio and far-infrared emission.  However, as discussed above, broadband intensity mapping correlation experiments are affected not only by luminosity correlations between different bands, but flux correlations as well. We thus attempt to quantify to what extent fluxes in two different bands may appear correlated due to distance effects even when their intrinsic luminosities are completely independent of each other. By distance effects we refer to to the effect that more distant objects are generally weaker in all bands than nearer objects. 

We first make a few approximations to get intuition for the effect, and then simulate the effect as a function of stricter and stricter flux cuts due to deeper and deeper foreground masking. Consider a sky survey over a fixed field of view, and a set of objects with uncorrelated infrared and radio emission. By uncorrelated we mean that the infrared and radio luminosities are independent random variables determined by the infrared and radio luminosities, respectively. Assume that the objects are uniformly distributed in space out to $z\sim0.5$, and assume cartesian space. We are interested in the effective correlation between radio and infrared fluxes in the same sky pixels, but let us approximate this by calculating the correlation between source fluxes in the two bands. Starting from Eqn. \ref{eqn:imagecorrdef}, we have
\begin{equation} % https://www.evernote.com/shard/s316/nl/2147483647/8270de10-6438-4d98-847a-a13e40ff9a14/
	c = \frac{\langle F_\rad F_\ir \rangle-\langle F_\rad\rangle\langle F_\ir\rangle}{\sqrt{(\langle F_\rad^2\rangle-\langle F_\rad\rangle^2)(\langle F_\ir^2\rangle-\langle F_\ir\rangle^2)}}
\end{equation}
Then writing this in terms of source fluxes $F_i=L_i/4\pi d^2$ for $i=\rad,\ir$, we find
\begin{equation}
	c = \frac{\beta-1}{\sqrt{(\beta\alpha_\rad-1)(\beta\alpha_\ir-1)}}
\end{equation}
where $\beta\equiv\langle d^{-4}\rangle/\langle d^{-2}\rangle^2$ and $\alpha_i=\langle L_i^2\rangle/\langle L_i\rangle^2$.
For a survey of a fixed angular field of view, uniform spatial distribution of objects, and cartesian spacetime, the distribution of source distances $d$ grows as $d^2$, which gives 
\begin{equation}
	\beta\approx\frac{d_\maxtext^3-d_\mintext^3}{3d_\maxtext d_\mintext (d_\maxtext-d_\mintext)}
\end{equation}
We observe that the radio--infrared flux correlation for some type of objects is a function of the radio and infrared luminosity function for that type of objects. In fact, we can see immediately that if the luminosity distributions are wide, their $\alpha$'s are large, and thus $c$ is small. Conversely, if the luminosity functions are narrow, then the distance to the sources plays a more significant role in determining the flux an object in question, and so $c$ is larger. 

\begin{figure*}[h]
\centering
\includegraphics[width=6in]{images/sim_rad_ir_luminosity_functions.pdf}
\caption{5' pixels}
\label{fig:luminosityfunctions}
\end{figure*}

To quantify whether this effect can explain our measured radio--infrared correlation in SF galaxies and the lack of one in AGN, we use AGN and SF 1.4\,GHz luminosity functions from \citet{mauch07} (Fig. \ref{fig:luminosityfunctions}, right panel) and AGN and SF and SF 8\,$\mu$m luminosity functions from \citet{fu10} (left panel). The former describe galaxies at $z<0.3$, while the latter describe galaxies at $z\sim0.6$. This analysis could be extended using proper redshift-dependent luminosity functions, possibly from simulations, though this anaylsis suffices to explain our earlier measurements and we leave a more detailed study for future work. We use approximately the same range of luminosities used by \citet{mauch07} and \citet{fu10}, though we adjust the minimum luminosities slightly to achieve the same number density of AGN ($\sim0.0020$/Mpc$^3$) and SF ($\sim0.00011$/Mpc$^3$) in both radio and infrared surveys. In the end we find that our results are only slightly sensitive to these luminosity minima as their faint ends become less and less significant in real surveys. 

We pick fiducial survey parameters of $d_\mintext=20$\,Mpc and $z_\maxtext=0.75$\,Mpc, giving $\beta\approx53.87$.  Using the above luminosity functions, we find $\alpha_{\text{SF,IR}}=1.474$, $\alpha_{\text{SF,rad}}=14.56$, $\alpha_{\text{AGN,IR}}=22.97$, $\alpha_{\text{AGN,rad}}=257.5$. These values agree with qualitative observation that the AGN luminosity function is wider than the SF luminosity function in both radio and infrared bands (Fig. \ref{fig:luminosityfunctions}). These values give a predicted radio--infrared correlation of 0.21 for SF and 0.01 for AGN agreeing with our finding of a significant radio--infrared correlation for SF and near-zero correlation for AGN. The exact values deviate from our measurements for a number of reasons. The MWA and WISE catalogs are not matched in depth or redshift coverage, and thus don't survey an exactly overlapping set of radio and infrared sources. Further, real world luminosity functions can exhibit significant redshift evolution. Of course, our measurement above did not even split up the MWA catalog into separate AGN and SF subsets as such detailed characterization of low frequency radio foregrounds remains an active area of research. 

BIG NOTE TO MYSELF ABOUT THIS COMPARISON OF SIMULATION TO DATA The average correlation over all $\ell$ (which is about 0.04) is pull down by the very small correlation large $\ell$ which is due to the low MWA resolution thus we should compare to the value at smaller $\ell$

\begin{figure*}[h]
\centering
\includegraphics[width=6in]{images/sim_correlation_agn_and_sf.pdf}
\caption{5' pixels}
\label{fig:simagnlfcorrelations}
\end{figure*}

Lastly, we ask to what extent this unwanted radio--infrared foreground correlation can be mitigated by masking the brightest sources. Using the luminosity functions discussed above, we simulate radio and infrared surveys for each of AGN and SF. We begin by generating the mock radio catalogs of AGN and SF, choosing a poisson random number of each in each of 400 logarithmic luminosity bins. We distribute the objects uniformly over a volume $D_\maxtext=cz_\maxtext/H_0=$3212\,Mpc deep and $\theta_{\text{FOV}}D_\maxtext=$1121\,Mpc wide. We then pick a random infrared luminosity for each radio object from the appropriate infrared luminosity function. Finally we plot in Fig. \ref{fig:simagnlfcorrelations}, along the lines of Fig. \ref{fig:correlationsandSNRs}, the 1.4\,GHz--8\,$\mu$m correlation of our mock AGN and SF catalogs after masking down to a maximum radio and infrared flux. As we saw above, without any flux cut we find a roughly 20\% radio--infrared correlation for AGN and only a percent level correlation for SF. As we mask fainter and fainter sources, the AGN correlation generally increases to the 5--10\% level, while the SF correlation first increases, then decreases after masking down to 10$^{-4}$\,Jy. With increasing mask depth, these correlations do not change monotonically and they do not generally approach zero, and more detailed modeling of effective foreground flux correlations will be necessary in real world intensity mapping correlation experiments probing the EOR. In the next section we move beyond the bright sources and study the magnitudes and correlation properties of the residual radio and infrared foregrounds in our MWA and ATLAS observations.

\section{Residual foregrounds and cross spectrum limits}

In this section we characterize the power spectra and correlation properties of the residual 185\,MHz and 850\,nm foregrounds after subtraction and masking of the the bright sources identified by the surveys discussed in the previous section.

\subsection{Residual 21\,cm foregrounds}

We use the broad band 185\,MHz images from the FHD pipeline summarized in Sec. \ref{sec:mwaobservations} to verify that the level of foreground residuals are as expected by comparison with 3D foreground power spectrum measurements. We also study how much observation time is required to achieve the best foreground subtraction with an eye towards the large survey areas required to mitigate the sample variance noise due to uncorrelated radio and infrared foregrounds in cross spectrum analyses. 

In fact, while 21\,cm intensity mapping measurements in 3D are limited by radio thermal noise and demand order thousand hour integrations to reveal the cosmological signal \citep{beardsley13,PoberNextGen}, noise in broadband measurements is subdominant to foreground residuals after much shorter integration times. Further, we expect to require large fields of view to recover any cosmological correlation between broadband 21\,cm and IR images. To that end, we examine whether all 32 hours are needed to achieve the lowest foreground residuals, or whether a subset will suffice. In fact, the $uv$ plane is essentially full after only 3 hours of rotation synthesis, and compounding days together serves only to increase signal to noise. 

The FHD outputs presented in Sec. \ref{sec:mwaobservations} are naturally weighted image space cubes of the raw data ($I_\text{nat}$), the model data ($I_\text{nat,mod}$), the weights ($I_w$) (i.e., the $uv$ sampling), and the primary beam. Each of these cubes has an \textit{odd} and \textit{even} version divided up at a 2\,sec cadence. We average all these cubes over frequency to make broad band images, then apply uniform weighting as
\begin{equation}
\label{eqn:uniformweighting}
I_\text{uni}(\vec{\theta}) = \frac{10^{-26}}{2k_B \lambda^2} \sum_{\vec{u}} \frac{\tilde{I}_\text{nat}(\vec{u})}{\tilde{I}_w(\vec{u})} e^{-2\pi i \vec{\theta}\cdot\vec{u}}d^2u
\end{equation}
$\tilde{I}_i(\vec{u}) = \sum_{\vec{\theta}} I_i(\vec{\theta}) e^{2\pi i\vec{\theta}\cdot\vec{u}} d\Omega$ for $i=\text{nat},w$.

We then compute the angular power spectrum as
\begin{equation}
	C(\ell)=\frac{\sum_{\vec{\ell}}|\tilde{I}_\text{uni}(\vec{\ell})\tilde{I}_w(\vec{\ell})|^2}{\sum_{\vec{\ell}}|\tilde{I}_w(\vec{\ell})|^2}
\end{equation}
noting that $\ell=2\pi u$, and the sum includes all $\vec{\ell}$ bins in the desired $\ell$ bin.  We estimate the thermal noise power spectrum by computing the power spectrum as above but using the odd minus even difference cube which contains only thermal noise.

\begin{figure}[h]
\centering
\includegraphics[width=3.5in]{images/res_pspec_of_100_obsids_with_diff_spacings_6amin.pdf}
\caption{the caption}
\label{fig:respspecspacingsstudy}
\end{figure}

We plot in Fig. \ref{fig:respspecspacingsstudy} the power spectra of the raw (i.e., un-foreground-subtracted image), residual, and noise images up to $\ell=2600$, corresponding to a maximum baseline length of $\sim$700\,m. Beyond this, the $uv$ coverage becomes quite sparse, introducing artifacts in the application of gridding and uniform weighting. We compare the power spectra of the deep 32 hour integration (black dashed) to those of 3 hour integrations spread over 1 (red), 5 (green), and 19 (blue) nights. Each of three three datasets consists of $\sim$100 two-minute integrations with a minimum spacing of 0, 5, and 24 minutes, all with occasional several day gaps due to observing constraints and quality cuts. 

We find that all four sets have the same raw power spectra, but the residual power spectrum decreases as the 3 hours are spread over more and more nights until it reaches the level of the deep 32 hour integration, a factor of $\sim2$ lower than that of the single night analysis. These findings are consistent with ionosphere-related errors, e.g., in interferometric calibration, whose ionospheric temporal correlations require many independent nights to average down. Thus while we use the 32 hour integration for convenience in this analysis, it is by no means necessary to use such a deep integration in a broad band analysis, especially when we will find later that survey speed is essential for such broad band correlation experiments. Note that as expected, the thermal noise of the deep integration is a factor of 10 below that of the 3\,hour integrations, which are themselves at least a factor of $\sim100$ below the fo reground residuals in these broad band images.

\subsection{Residual IR foregrounds}
\label{sec:resirfg}

We proceed to generate foreground-subtracted 850\,nm images of the four deep ATLAS integration fields shown in Fig. \ref{fig:surveyoverview}. Each of these fields is a stack of 9 30\,sec exposures with 5$^\circ$ field of view, dithered such that the overlap is a $\sim4^\circ$ region. By confining ourselves to this overlap region, we avoid the background discontinuities which affect many nominally wide field infrared image datasets whose mosaicing introduces significant background patchiness. 

Our approach is to mask the high resolution image at 1.86'' resolution, then coarse grid down to 3.5', of order the resolution of the MWA, taking each coarse pixel's value as the average of all unmasked fine pixels within it. If fewer than 10\% of its fine pixels remain unmasked, we consider the whole coarse pixel masked to avoid introducing too much noise variation over coarse pixels. We apply the high resolution source masking in the following four stages, which we illustrate in Fig. \ref{fig:bigfgmaskingstudy} for the ATLAS field centered at (RA,Dec)$=2.74^\circ, -24.79^\circ$) (the top right red box in Fig. \ref{fig:surveyoverview}). Each row shows the result of an additional masking stage, as outlined below. The left column shows the 9' field at the center to inspect the masking close in; the center column shows the coarse binned image with 3.5' resolution; and the right column shows the FFT of the center image (plotted as $\log \Delta(\vec{\ell})/(\text{kJy/sr})$) in order to identify detector systematics. 

\begin{figure*}[h]
\centering
\includegraphics[width=7in]{images/big_foreground_masking_study_2.pdf}
\caption{aoeu}
\label{fig:bigfgmaskingstudy}
\end{figure*}

\begin{enumerate}
	\item \textbf{Mask saturated regions.} (Fig. \ref{fig:bigfgmaskingstudy}, row 1) Nearly saturated pixels are associated with nearby bright stars which would dominate fluctuation measurements, we thus mask 4' around all pixels within 30\% of saturation (white regions in left and center columns). On average over our fields, 92\% of fine pixels remain unmasked after this step, and 96\% of coarse pixels remain.
	\item \textbf{Mask sources to 5$\sigma$.} (Fig. \ref{fig:bigfgmaskingstudy}, row 2) We mask circular regions with radius equal to five times the profile RMS along the minor axes of each source as measured by SExtractor (see Sec. \ref{sec:catalogs}). The reason we use the minor axis RMS is that vertical charge leakage in the CCD array results in unrealistically large major axis RMS measurements for bright sources. After this stage $\sim87$\% of fine pixels remain unmasked, and the fraction of coarse pixels remaining is unaffected.
	\item \textbf{Mask sources to $12\sigma$ and other emission above 100\,kJy/sr.} (Fig. \ref{fig:bigfgmaskingstudy}, row 3) We use a larger masking radius to remove the remaining diffuse halos around the 5$\sigma$ source masks due to the finite point spread function. We also determine that the vertical CCD charge leakage can be isolated by looking for emission brighter that 100\,kJy, and that it can be flagged without cutting into the shot noise between sources. 58\% of fine pixels remain unmasked after this step, and again the fraction of coarse pixels remaining is unaffect.d
	\item \textbf{Mask fourier modes matching $|\ell_x|<200$ or $|\ell_y|<200$.} (Fig. \ref{fig:bigfgmaskingstudy}, row 4) The previous two stages revealed a detector artifact within $\Delta\ell sim100$ of $\ell_x=0$ and $\ell_y=0$. These compact fourier systematics correspond to slight horizontal and vertical discontinuities in the center image due to imperfect gain matching between 16 different detector amplifiers servicing different regions of the CCD array. We conservatively mask all modes within $\Delta\ell=200$ of either of these axes to eliminate this effect.
\end{enumerate}

Note that these 850\,nm deep observations were recorded during near new moon conditions, and we find that the mean air glow in source-free regions is $\sim3\times10^3$ kJy/sr, of order 19 AB mag/arcsec$^2$. \citet{sullivan12} measure a 1020\,nm continuum air glow brightness (i.e., after spectrally masking the OH lines) of $20\pm0.5$ AB mag/asec$^2$ far away from the moon. 

We proceed to characterizing the residual infrared foreground fluctuations in power spectrum space, using the optimal quadratic estimator to properly account for the masking of coarse pixels. This estimator was introduced by \citet{Maxpowerspeclossless} to recover power spectra of CMB maps with arbitrary survey geometries and noise properties, and have recently been revived for power spectrum analysis of 21\,cm data by \citet{X13, dillonneben, LT11, DillonFast, ali15}. We employ this estimator in a manner more similar to the original CMB case, using it to account for pixel masking in 2D power spectrum estimation. We briefly summarize the estimator here, and refer to \citet{X13} for a more detailed description.

We label the normalized estimator of the power in bin $\alpha$ as $p_\alpha$, related to the unnormalized estimator $q_\alpha$ as $\mathbf{p} = \Mb \mathbf{q}$. The unnormalized estimator is given by
\begin{equation}
q_\alpha = \frac{1}{2}(\xb-\langle\xb\rangle)^\dagger \Cb^{-1} C_{,\alpha}\Cb^{-1}(\xb-\langle\xb\rangle)
\end{equation}
where $\xb$ is a column vector containing all $N$ pixel measurements, $\Cb$ is the pixel-pixel covariance matrix, and $C_{,\alpha}=d\Cb/dp_\alpha$ is the derivative of the covariance with respect to the power in bin $\alpha$. Note that $C_{,\alpha} = \Ab^\dagger\Ab$, where $\Ab$ is a $N_\alpha\times N$ with elements $A_{ij}=\exp(i\vec{\theta}_j\cdot\vec{\ell}_i)$. Here $\vec{\ell}_j$ refers to the $j$'th $\vec{\ell}$ mode in bin $\alpha$. 

The matrix of window functions (i.e., horizontal error bars) of the band powers $p_\alpha$, defined such that $\pb_\text{estimated}=\Wb\pb_\text{true}$ is given by $\Wb=\Mb\Fb$, where $\Fb$ is the Fisher matrix. The covariance between the measured $p_\alpha$ values is given by $\mathbf{\Sigma} = \Mb\Fb\Mb^t$. \citet{X13} argue that taking $\Mb\propto \Fb^{-1/2}$ is a good compromise between wanting small horizontal error bars and small vertical error bars. For simplicity, we normalize our estimates in an additional step at the end, taking $\Mb=\Fb^{-1/2}$, dividing each element of $\pb$ by the peak of the appropriate row of $\Wb$. Lastly, the elements of the Fisher matrix are given by
\begin{equation}
\Fb_{\alpha\beta}=\frac{1}{2}\text{tr}\left(\Cb^{-1} C_{,\alpha} \Cb^{-1} C_{,\beta} \right)	
\end{equation}

Now we turn to application of this formalism to power spectrum estimation from our maskeod IR images. Later we will adapt it to estimation of the 21\,cm--IR cross spectrum. We take $\xb$ to be a vector of all IR coarse (3.5') pixel values, with masked pixels set to zero. After gridding the high resolution images by a factor of $\sim10^4$ to reach this resolution, photon shot noise is negligible, and the covariance is the sum of the signal covariance $\Cb_\text{signal}$ and the masking covariance $\Cb_\text{mask}$. $\Cb_\text{mask}$ is a diagonal matrix with $\infty$ for masked pixels, and 0 otherwise. In practice, we replace $\infty$ with a number $10^10$ times larger than all eigenvalues of $\Cb_\text{signal}$, finding that the results are not sensitive to this parameter. The signal covariance is easily obtained by writing it in fourier space, where it is a diagonal matrix with a guess of the true power spectrum on the diagonal, then fourier transforming it into image space with fourier transform matrices $\mathcal{F}$. 
\begin{equation}
\label{eqn:covFTwithmask}
	\Cb = \mathcal{F}^\dagger\Cb_\text{ft}\mathcal{F}+\Cb_\text{mask}
\end{equation}
Note that $\mathcal{F}$ is an $N\times N$ matrix with elements $\mathcal{F}_{ij}=\exp(-i \vec{\theta}_i\cdot\vec{\ell}_j)$, where $i$ runs over all pixels and $j$ runs over all fourier cells. 

\begin{figure}[h]
\centering
\includegraphics[width=3.5in]{images/big_foreground_masking_study_pspecs_2.pdf}
\caption{WHAT ABOUT THE UNCERTAINTY IN THE ATLAS ABSOLUTE CALIBRATION ?????}
\label{fig:bigfgmaskingstudypspecs}
\end{figure}

Using this estimator, we calculate the power spectrum of all four deep ATLAS fields after each stage of masking, and plot the results in Fig. \ref{fig:bigfgmaskingstudypspecs}. The power spectrum of all 850\,nm sources, masking only saturated regions rises as $\ell^1$ (red dots connected by red dotted lines), as expected when plotting $\Delta=\sqrt{\ell^2C_\ell/2\pi}$. These red curves vary by $\pm50\%$ between the different fields. Masking sources out to $5\sigma$ removes of order two orders of magnitude in power (cyan dots connected by cyan dashed lines), and masking out to $12\sigma$ and above 100\,kJy/sr removes a factor of a few more in power (yellow dots connected by solid yellow lines). 

Lastly, excluding modes with $|\ell_x|<200$ or $|\ell_y|<200$ results in a spectrum (blue squares connected by solid blue lines) consistent in many ways with CIBER's 1.1\,$\mu$m results (black dots) \citep{zemcov14}. We first note that increasing any of the masking parameters introduced here in this section does not significantly alter these results. Increasing the masking radius around saturated pixels, for instance, results in masking of more coarse pixels, but removes negligible additional power from these residual spectra. Our final residual power spectrum agrees well with the CIBER measurements at $\ell\lesssim1000$, above which it continues to closely track the CIBER measurements but with a factor of $\sim2$ larger amplitude. It is also compelling that only our final residual spectrum agrees to better than 10\% in power over all four fields. They agree less precisely at low $\ell$ as expected due to the larger sample variance noise due to the logarithmic binning and the fewer $\vec{\ell}$ cells available. 

\citet{zemcov14} compare the power averaged over $500<\ell<2000$ in their two bands (1.1\,$\mu$m and 1.5\,$\mu$m, each with 0.5\,$\mu$m bandwidth) with longer wavelength measurements up to 4.5\,$\mu$m by \citet{cooray12,kash3,matsumoto11}. They find that all these measurements fit a Rayleigh-Jeans spectrum, and it is unknown far into the near infrared or optical band the anisotropy spectrum continues to rise before turning over. We integrate the residual fluctuation power in our $850\pm75$\,nm band over this range in $\ell$ and find an RMS fluctuation power of $\langle\Delta(\ell)\rangle_{500<\ell<2000}=0.90\pm0.03$\,kJy/sr \footnote{These units result from taking the intensity field to be $I_f$.}, equal to $3.2\pm0.1$\,nW/m$^2$/sr \footnote{These units result from taking the intensity field to be $\lambda I_\lambda$, as in \citet{zemcov14}.}. This measurement deviates at the $20\sigma$ level from the 850\,nm Rayleigh-Jeans prediction of 5.5\,nW/m$^2$/sr, indicating that we are likely starting to see the spectrum turn over. \citet{zemcov14} observed weak evidence for this with their measurements at 1.1\,$\mu$m and 1.6\,$\mu$m, but their smaller field of view ($\sim1$ deg sq.) limited their sensitivity to these angular scales, and their measurements deviate from the Rayleigh-Jeans prediction at only the 1$\sigma$ level. 

\subsection{Modeling the 21\,cm--Ly$\alpha$ cross spectrum}

\begin{figure*}[h]
\centering
\includegraphics[width=7in]{images/spectra3D_to_2D.pdf}
\caption{oeu}
\label{fig:spectra3Dto2D}
\end{figure*}

For comparison, we generate optimistic and pessimistic theoretical cross spectra. We simulate 21\,cm and Ly-$\alpha$ cubes taking 21\,cm power spectra from \citet{PoberNextGen}, Ly-$\alpha$ power spectra from \cite{Gong2014}, and the coherence between the two fields from \citetext{Heneka2016}. Combining simulations from all these sources allows us to better estimate the modeling uncertainty. Future work is needed to more self-consistently model these fields and their correlation over a range of possible reionization scenarios. From these power spectra and coherence functions, we generate approximate 21\,cm and Ly-$\alpha$ cubes assuming gaussian statistics. The simulated cubes have (1\,Mpc)$^3$ resolution over a (218\,Mpc)$^3$ volume at $z=7$, corresponding to $\Delta z=0.6$ and 0.4' angular resolution. The 21\,cm and Ly-$\alpha$ cubes have units of mK and kJy/sr, respectively, and we average them in the line of sight direction to produce broad band images. 

\citet{Gong2014} model the Ly-$\alpha$ cross spectrum and plot an uncertainty region over a range of likely values of   escape fraction of ionizing photons, fraction of radiation emitted at Ly$\alpha$, star-forming rate, and IGM clumping factor (their Fig. 1). We take the upper and lower edges of their uncertainty region as our optimistic and pessimistic power spectra, respectively. 

Similarly, \citet{PoberNextGen} simulate 21\,cm power spectra over a range of reionization scenarios with various values of ionizing efficiency ($\zeta$) (including the escape fraction), the minimum virial temperature of halos ($T_\text{ir}$) producing significant ultraviolet photons, and their mean free path in the IGM ($R_\text{mfp}$). Whether the signal at our redshift of interest ($z=7$) depends largely on whether reionization is already largely finished by that time or not. So we take as our pessimistic scenario the model with $z=8$ power spectrum of the ($\zeta =31.5$, $T_\text{vir}=1.5\times10^4$\,K, $R_\text{mfp}=30$\,Mpc), whose reionization midpoint is $z=9.5$. We take as our optimistic model a late reionization scenario the model with $T_\text{vir}=3\times10^5$\,K (other parameters unchanged), whose reionization midpoint is $z=5.5$. 

We plot the original 3D power spectra and coherence function in top panel of Fig. \ref{fig:spectra3Dto2D}, and the computed 2D power spectra from the line-of-sight averaged cubes, as well as their coherence, in the bottom panel. The 3D power spectra are plotted as $\Delta(k)=\sqrt{P(k)k^3/2\pi^2}$, while the 2D spectra are plotted as $\Delta(\ell)=\sqrt{C_\ell \ell^2/2\pi}$. As expected, line of sight averaging tends to remove power on short spatial scales (compared the line of sight depth of the cube), but it acts similarly on both cubes, so the coherence function is preserved. Note that the cross spectrum $\Delta_{12}$ is defined in terms of the coherence as $\Delta_{12}=c\sqrt{\Delta_1\Delta_2}$, and in the next section when we calculate the cross spectra, we assume the same coherence function from \citet{Heneka2016} in optimistic and pessimistic scenarios.

\subsection{Limits on the 21\,cm--Ly$\alpha$ cross spectrum}

Having characterized the residual sky power spectrum at 185\,MHz and 850\,nm after the best foreground masking and subtraction permitted by our datasets, we now search for a correlation between these radio and foreground residuals. To account for the non-uniform $uv$ sampling of the MWA image as well as the image space masking of the ATLAS image, we again use the optimal quadratic estimator. In Appendix \ref{sec:optimalestimatorforcrossspectrum} we show that with the approximation that the correlation between the two images is small, the proper extension of the optimal quadratic estimator from the auto spectrum case to the cross spectrum case is given by
\begin{equation}
q_\alpha \approx (\xb_{21}-\langle\xb_{21}\rangle)^\dagger \Cb_{21}^{-1} \Cb_{,\alpha}\Cb_\IR^{-1}(\xb_\IR-\langle\xb_\IR\rangle)
\end{equation}
\begin{equation}
\Fb_{\alpha\beta}\approx\text{tr}\left(\Cb_{21}^{-1} \Cb_{,\alpha} \Cb_\IR^{-1}  \Cb_{,\alpha}  \right)	
\end{equation}
where $\Cb_{,\alpha}$ is the same matrix used in Sec. \ref{sec:resirfg}, $\mathbf{x}_{21}$ and $\mathbf{x}_{IR}$ are the vectors of 21\,cm and IR pixel values, respectively, and $\mathbf{w}_{21}$ is the vector of the image space radio point spread function (i.e., synthesized beam). As before we use $\Mb=F^{-1/2}$ to normalize the estimator in  $\pb=\Mb\qb$, and perform a final normalization using the peaks of the window functions $\Wb=\Mb\Fb$. We use the same IR covariance matrix given in Eqn. \ref{eqn:covFTwithmask}, and generate the 21\,cm covariance matrix by writing the diagonal thermal noise covariance matrix fourier space, then fourier transforming it as
\begin{equation}
\label{eqn:C21}
\Cb_{21} = \mathcal{F}^\dagger\text{diag}(|\mathcal{F}\mathbf{w}_{21}|)\mathcal{F}
\end{equation}
Though Fig. \ref{fig:respspecspacingsstudy} shows that thermal noise in these broad band images is quite subdominant to the foreground residuals over all $\ell$, it is still beneficial to downweight the most poorly sampled $\vec{\ell}$ modes to mitigate $uv$ gridding artifacts. 

We apply this formalism to each of the four $4^\circ$ deep ATLAS fields shown in Fig. \ref{fig:surveyoverview}, pairing each IR image with the overlapping region of the MWA image cropped from the naturally weighted image. We crop the image space synthesized beam over the same field of view, use it to apply uniform weighting to the cropped MWA image using Eqn. \ref{eqn:uniformweighting}, and use it again to construct $\Cb_{21}$ in Eqn. \ref{eqn:C21}. 

\begin{figure}[h]
\centering
\includegraphics[width=3.6in]{images/mwa_atlas_xspec_with_2Dsimtheory_and_2sigma_errors.pdf}
\caption{oeu}
\label{fig:resxspec}
\end{figure}

In Fig. \ref{fig:resxspec} we plot the resulting cross spectrum (red markers) averaged over all four fields with $2\sigma$ error bars. The lowest two $\ell$ modes are marginal detections, and the rest are consistent with zero. Roughly half the estimated bandpowers are negative (triangles) and positive (circles), as expected if there is no measured correlation.  Our bins are evenly spaced in $\log \ell$, and it is beneficial to overlap them slightly since our normalization matrix $M\sim F^{-1/2}$ ensures bin errors are always uncorrlated.  We interpret the tops of the error bars as 95\% upper limits on the cosmological cross spectrum of 21\,cm and Ly$\alpha$ emission from the EOR, achieving a strongest upper limit of $\Delta_{21,\text{Ly}\alpha}<14\,(\text{kJy/sr}\cdot \text{mK})^{1/2}$ (95\%). 

The cross spectrum uncertainty due to sample variance noise of residual foregrounds scales approximately as $d\Delta_{21,\IR}\propto f_{21\text{FG}}^{1/2}d\theta/\theta_{\text{FOV}}$, where $f_{21,\text{FG}}$ is the fraction of remaining 21\,cm foreground power, $d\theta$ is the pixel resolution, and $\theta_{\text{FOV}}$ is the field of view. Our MWA/ATLAS experiment has ($f_{21,FG}\approx0.1$, $d\theta=3'$, $\theta_{\text{FOV}}=8^\circ$). We plot the predicted sensitivities for improved experiments with (0.1, 0.3', 8$^\circ$) (solid blue), (0.01, 0.3', 8$^\circ$) (dashed blue), and (0.01, 0.3', 40$^\circ$) (dotted blue). \citet{zemcov14,cooray12} argue that the remaining IR foreground is largely due to intrahalo light around already masked sources, and cannot be masked without removing essentially all pixels in the image. We thus do not assume that IR foreground subtraction can be improved.

We find that after increasing the image resolution (of the 21\,cm image) by $20\times$ (to 0.3') and reducing its fractional foreground residuals by $10\times$ (to 0.01), a tenuous detection of the optimistic 21\,cm--Ly$\alpha$ cross spectrum is possible at $\ell<70$. Widening the field of view (of the IR image) by a factor of $5\times$ (to $40^\circ$) permits significant detections at $\ell<1500$. In all these cases, the cross spectrum remains out of reach in the pessimistic scenario, though ruling out the optimistic scenario would significantly constrain reionization models.

\section{Discussion}


\begin{acknowledgments}
We acknowledge helpful discussions on optimal quadratic estimators with Adrian Liu, Josh Dillon, and Aaron Ewall-Wice. 
Our analysis code is publicly available at \ULurl{https://github.com/abrahamneben/21cmIRxcor}
\end{acknowledgments}

\appendix

\section{Power spectrum of photon shot noise}
\label{sec:Pshot}

In Sec. \ref{sec:pspecIRfg} we measure the maximum airglow to be $I_\text{air}=5\times10^3$ kJy/sr, and in this appendix we calculate the power spectrum of this photon shot noise. We must observe that the mean number of photons collected by a pixel during each observation is $\langle N_\text{ph}\rangle=I_\text{air}At_\text{int} \Delta f d\theta^2/hf$, where $A=(0.5\,\text{m})^2$ is the collecting area of ATLAS, $t_\text{int}=30\,$sec, $\Delta f$ and $f$ are the frequency bandwidth and center frequency of I band, and $d\theta$ is the pixel size. The passband has $\Delta\lambda=150\,$nm and $\lambda=800\,$nm. 

The shot noise contribution to the power spectrum is given by
\begin{equation}
C_{\IR, \shot}(\vec{\ell}) = \left\langle\left|\sum_{m,n}I_\shot(m,n)e^{-2\pi i(ma+nb)/N}\right|^2\right\rangle \frac{d\theta^2}{N^2}
\end{equation}
where $I_\shot(m,n)\equiv I(m,n)-\langle I(m,n)\rangle$ denotes the photon shot noise contribution to pixel (m,n), and $N$ is the number of pixels on each side of the square image. Then using the fact that the shot noise is uncorrelated between different pixels, we find
\begin{equation}
C_{\IR, \shot}(\vec{\ell}) = \sum_{m,n}\langle I^2_\shot(m,n)\rangle \frac{d\theta^2}{N^2}
\end{equation}
Note that $I(m,n)=N_\text{ph}(m,n)hf/\Delta f A t_\text{int}d\theta^2$ and $\langle N_\text{ph}^2\rangle = \langle N_\text{ph}\rangle$, so we have
\begin{equation}
C_{\IR, \shot}(\vec{\ell}) = \langle N_\text{ph}\rangle \left(\frac{hf}{\Delta f A t_\text{int}d\theta^2}\right)^2 d\theta^2
\end{equation}

\begin{equation}
C_{\IR, \shot}(\vec{\ell}) =\frac{I_\text{air}h\lambda}{\Delta \lambda A t_\text{int}}
\end{equation}

\section{Relation between the power spectrum of image cubes and broadband images}
\label{sec:pspecrelation}

We focus in this paper on the spherical power spectrum of broadband images, $C_\ell$,  instead of that of image cubes, $P(\vec{k})$, as 21\,cm observations have focused on. Here we work out the approximate relation between the two over small fields of view (i.e., for large $\ell$) to facilitate comparison with past 21\,cm power spectrum results. In particular, we calculate the scaling factor $B$ relating the purely transverse modes of the power spectrum $P(k_\perp,k_\parallel=0)$ of a image cube $I(\theta_x,\theta_y,f)$ to the spherical power spectrum of a broad band image $C_\ell$ as
\begin{equation}
P(k_\perp,k_\parallel=0) = B C_{\ell(k_\perp)}
\end{equation}

Using the fourier transform convention discussed in Sec. \ref{sec:pspecconventions}, the left side of the equation is given by
\begin{equation}
P(k_\perp,k_\parallel=0) = \frac{1}{N_\perp^2 N_\parallel dV}\langle|\tilde{I}(k_x,k_y,k_\parallel=0)|^2\rangle
\end{equation}
where $N_\perp\equiv N_x=N_y$ is the number of pixels in each of the two transverse dimensions of the image cube, and $N_\parallel$ is the number of pixels in the line of sight (ie, frequency) dimension. The comoving pixel volume is $dV = (D_c d\theta)^2 (\Delta D_c/N_\parallel)$, where $D_c$ is the line of sight comoving distance from the present day to the center of the cube, and $\Delta D_c$ is the comoving line of sight thickness of the cube. Lastly, recall that $k_\perp$ is related to $k_x$ and $k_y$ as $k_\perp=\sqrt{k_x^2+k_y^2}$.

Now substituting the definition of the fourier transform, we find

\begin{equation}
P(k_\perp,k_\parallel=0) =\frac{1}{N_\perp^2 N_\parallel dV}\left\langle\left|dV\sum_{\theta_x,\theta_y,f}I(\theta_x,\theta_y,f)e^{iD_c(k_x\theta_x+k_y\theta_y)}\right|^2\right\rangle
\end{equation}

Simplifying and writing this in terms of the broadband image $I_{\Delta f}(\theta_x,\theta_y)\equiv\frac{1}{N_\parallel}\sum_f  I(\theta_x,\theta_y,f)$, we find

\begin{equation}
P(k_\perp,k_\parallel=0) =(D_c^2 \Delta D_c)
\frac{d\theta^2}{N_\perp^2}\left\langle\left|\sum_{\theta_x,\theta_y}I_{\Delta f}(\theta_x,\theta_y)e^{iD_c(k_x\theta_x+k_y\theta_y)}\right|^2\right\rangle
\end{equation}

Now denote $k_x=a\cdot dk$, $k_y=b\cdot dk$, $\theta_x=m\cdot d\theta$, and $\theta_y=n\cdot d\theta$, where $dk = 1/N_\perp D_c d\theta$. 

\begin{equation}
P(k_\perp(\ell(a,b)),k_\parallel=0) =(D_c^2 \Delta D_c)
\frac{d\theta^2}{N_\perp^2}\left\langle\left|\sum_{m,n}I_{\Delta f}(m,n)e^{2\pi i(am + bn)/N_\perp}\right|^2\right\rangle
\end{equation}

Comparing with Equations \ref{eqn:Cldef}, \ref{eqn:elldef}, and \ref{eqn:elldef2}, we see that $B\equiv P(k_\perp,k_\parallel=0)/ C_{\ell(k_\perp)}=D_c^2 \Delta D_c$ and $\ell=D_c k_\perp$.


\section{Extending the optimal quadratic power spectrum estimator to the cross spectrum case}
\label{sec:optimalestimatorforcrossspectrum}

The optimal quadratic estimator formalism presented in Sec. \ref{sec:resirfg} was constructed to estimate the power spectrum of an image with arbitrary pixel sampling and noise properties. In this section we extend this formalism to achieve the same advantages in cross spectrum measurements. 

Consider measurements at two bands over the same set of pixels on the sky, $\xb_1$ and $\xb_2$, each column vectors with $N$ elements. Let us combine these together into a single column vector containing all measurements as $\xb=\left(\begin{matrix}\xb_1 \\ \xb_2  \end{matrix}\right)$, whose covariance is given by 

\begin{equation}
\Cb\equiv  \langle\xb\xb^\dagger\rangle-\langle\xb\rangle\langle\xb^\dagger\rangle=\left(\begin{matrix}\Cb_1 & \Cb_{12} \\ \Cb_{12}^\dagger & \Cb_2   \end{matrix}\right)
\end{equation}
and 
\begin{equation}
\frac{d\Cb}{dp_{12}^\alpha}=\left(\begin{matrix}\mathbf{0} & \Cb_{,\alpha}\\ \Cb_{,\alpha}^\dagger & \mathbf{0}   \end{matrix}\right)
\end{equation}
$\Cb_1$ and $\Cb_2$ depend only on the auto power spectra of the different fields; only $\Cb_{12}$ depends on the cross spectrum. Said another way, $C_{,\alpha}$ is the same matrix used in Sec. \ref{sec:resirfg}, but used here in the off-diagonal parts of $d\Cb/dp_{12}^\alpha$ so as to capture the cross products between the two fields\footnote{One might object to our form of $d\Cb/dp_{12}^\alpha$, arguing that artificially limiting ourselves to cross products between the two fields is tantamount to throwing a significant amount of the information contained in the data sets, and thus our estimator cannot be optimal. There are certainly situations in which this would be the case. If we had some theory of how each field was related to the matter density field of the universe, then both the auto-products and cross-products contain similar information. However we take an empirical approach where we assume we know nothing about either field but want to know their correlation. Only the cross products contain that information.}. And as before, the unnormalized estimator $q_\alpha$ of the power in band $\alpha$ is given by
\begin{equation}
q_\alpha = \frac{1}{2}(\xb-\langle\xb\rangle)^\dagger \Cb^{-1} \frac{d\Cb}{dp_{12}^\alpha}\Cb^{-1}(\xb-\langle\xb\rangle)
\end{equation}
and the elements of the Fisher matrix are
\begin{equation}
\Fb_{\alpha\beta}=\frac{1}{2}\text{tr}\left(\Cb^{-1} \frac{d\Cb}{dp_{12}^\alpha} \Cb^{-1}  \frac{d\Cb}{dp_{12}^\beta}  \right)	
\end{equation}

In the case where the correlation between the two fields is expected to be weak, and we are primarily interested in setting an upper limit, we can get a significant speedup by approximating $\Cb_{12}\approx0$ in our guess covariance. The expressions for $q_\alpha$ and $F_{\alpha\beta}$ then simplify to:
\begin{equation}
q_\alpha \approx (\xb_1-\langle\xb_1\rangle)^\dagger \Cb_1^{-1} \Cb_{,\alpha}\Cb_2^{-1}(\xb_2-\langle\xb_2\rangle)
\end{equation}
\begin{equation}
\Fb_{\alpha\beta}\approx\text{tr}\left(\Cb_1^{-1} \Cb_{,\alpha} \Cb_2^{-1}  \Cb_{,\alpha}  \right)	
\end{equation}


%% The reference list follows the main body and any appendices.
%% Use LaTeX's thebibliography environment to mark up your reference list.
%% Note \begin{thebibliography} is followed by an empty set of
%% curly braces.  If you forget this, LaTeX will generate the error
%% "Perhaps a missing \item?".
%%
%% thebibliography produces citations in the text using \bibitem-\cite
%% cross-referencing. Each reference is preceded by a
%% \bibitem command that defines in curly braces the KEY that corresponds
%% to the KEY in the \cite commands (see the first section above).
%% Make sure that you provide a unique KEY for every \bibitem or else the
%% paper will not LaTeX. The square brackets should contain
%% the citation text that LaTeX will insert in
%% place of the \cite commands.

%% We have used macros to produce journal name abbreviations.
%% AASTeX provides a number of these for the more frequently-cited journals.
%% See the Author Guide for a list of them.

%% Note that the style of the \bibitem labels (in []) is slightly
%% different from previous examples.  The natbib system solves a host
%% of citation expression problems, but it is necessary to clearly
%% delimit the year from the author name used in the citation.
%% See the natbib documentation for more details and options.

%\bibliography{xcor_paper}
\bibliography{xcor_paper_new}


\end{document}
